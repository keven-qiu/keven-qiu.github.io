\documentclass[12pt]{article}
% \documentclass[12pt]{report}

\setlength{\parskip}{1.5ex}

\usepackage[utf8]{inputenc}
\usepackage[T1]{fontenc}
\usepackage{lmodern}

\usepackage[margin=1in]{geometry}
\usepackage{amsmath,amssymb,amsthm,amsfonts,enumerate,nicefrac,hyperref,fancyhdr,diffcoeff,tabularx}
\usepackage{algorithm,algorithmicx}
\usepackage[noend]{algpseudocode}
\usepackage{mathtools}
\usepackage{graphicx}
\usepackage[shortlabels]{enumitem}
\usepackage[most]{tcolorbox}
\usepackage{tikz}
\newcommand{\bb}[1]{\mathbb{#1}}
\newcommand{\mc}[1]{\mathcal{#1}}
\newcommand{\tbf}[1]{\textbf{#1}}
\newcommand{\ds}{\displaystyle}
\def\F{{\mathbb{F}}}
\def\Q{{\mathbb{Q}}}
\def\Z{{\mathbb{Z}}}
\def\N{{\mathbb{N}}}
\def\R{{\mathbb{R}}}
\def\K{{\mathbb{K}}}
\def\C{{\mathbb{C}}}
\def\A{{\mathbb{A}}}
\def\P{{\mathbb{P}}}
\def\E{{\mathbb{E}}}
\def\V{{\mathbb{V}}}
% \newcommand{\df}[1]{\textbf{Def #1}:}
% \newcommand{\thm}[1]{\textbf{Theorem #1}}
% \newcommand{\cor}[1]{\textbf{Corollary #1}}
% \newcommand{\pp}[1]{\textbf{Proposition #1}}
% \newcommand{\lma}[1]{\textbf{Lemma #1}}
\newcommand{\prf}{\textbf{\textit{Proof. }}}
\newcommand{\mbf}[1]{\mathbf{#1}}
\newcommand{\mrm}[1]{\mathrm{#1}}
\newcommand{\ip}[1]{\left\langle#1\right\rangle}
\newcommand{\abs}[1]{\left\lvert#1\right\rvert}
\newcommand{\norm}[1]{\left\lVert#1\right\rVert}
\newcommand{\ceil}[1]{\left\lceil#1\right\rceil}
\newcommand{\floor}[1]{\left\lfloor#1\right\rfloor}
\newcommand{\ol}[1]{\overline{#1}}
\newcommand{\ul}[1]{\underline{#1}}
\newcommand{\ob}[1]{\overbrace{#1}}
\newcommand{\ub}[1]{\underbrace{#1}}
\newcommand{\antidiff}[4]{\int\limits_{#1}^{#2}#3\,d#4}
\newcommand{\antiidiff}[5]{\iint\limits_{#1}^{#2}#3\,d#4\,d#5}
\newcommand{\antiiidiff}[6]{\iiint\limits_{#1}^{#2}#3\,d#4\,d#5\,d#6}
\newcommand{\opname}[1]{\operatorname{#1}}
\newcommand{\pmat}[1]{\ensuremath{\begin{pmatrix} #1 \end{pmatrix}}}
\newcommand{\bmat}[1]{\ensuremath{\begin{bmatrix} #1 \end{bmatrix}}}
\newcommand{\qbinom}[2]{\genfrac[]{0pt}{}{#1}{#2}_q}
\newcommand{\stirling}[2]{\genfrac\{\}{0pt}{}{#1}{#2}}
\newcommand{\paren}[1]{\left(#1\right)}
\newcommand{\sbrack}[1]{\left[#1\right]}
\newcommand{\cbrack}[1]{\left\{#1\right\}}
\newcommand{\st}{\text{s.t.}}
\DeclareMathOperator*{\argmax}{arg\,max}
\DeclareMathOperator*{\argmin}{arg\,min}
\DeclareMathOperator*{\rank}{rank}
\DeclareMathOperator*{\oc}{oc}
\DeclareMathOperator*{\defic}{def}

\renewcommand{\qed}{\hfill$\blacksquare$}

\newcommand{\tbfbox}[2]{\begin{tcolorbox}[title=#1,colback=black!5!white,colbacktitle=black!15!white,coltitle=black,boxrule=0pt,fonttitle=\bfseries,rightrule=0pt,bottomrule=0pt,toprule=0pt,leftrule=1mm]#2\end{tcolorbox}}
\newcommand{\df}[2]{\begin{tcolorbox}[title=Definition: #1,colback=cyan!10!white,colbacktitle=cyan!35!white,coltitle=black,boxrule=0pt,fonttitle=\bfseries,rightrule=0pt,bottomrule=0pt,toprule=0pt,leftrule=1mm,colframe=cyan!80!black]#2\end{tcolorbox}}
\newcommand{\pp}[2]{\begin{tcolorbox}[title=Proposition #1,colback=green!10!white,colbacktitle=green!30!white,coltitle=black,boxrule=0pt,fonttitle=\bfseries,rightrule=0pt,bottomrule=0pt,toprule=0pt,leftrule=1mm,colframe=green!80!black]#2\end{tcolorbox}}
\newcommand{\thm}[2]{\begin{tcolorbox}[title=Theorem #1,colback=red!10!white,colbacktitle=red!30!white,coltitle=black,boxrule=0pt,fonttitle=\bfseries,rightrule=0pt,bottomrule=0pt,toprule=0pt,leftrule=1mm,colframe=red!80!black]#2\end{tcolorbox}}
\newcommand{\lma}[2]{\begin{tcolorbox}[title=Lemma #1,colback=red!10!white,colbacktitle=red!30!white,coltitle=black,boxrule=0pt,fonttitle=\bfseries,rightrule=0pt,bottomrule=0pt,toprule=0pt,leftrule=1mm,colframe=red!80!black]#2\end{tcolorbox}}
\newcommand{\cor}[2]{\begin{tcolorbox}[title=Corollary #1,colback=violet!10!white,colbacktitle=violet!25!white,coltitle=black,boxrule=0pt,fonttitle=\bfseries,rightrule=0pt,bottomrule=0pt,toprule=0pt,leftrule=1mm,colframe=violet!80!black]#2\end{tcolorbox}}

\begin{document}
% Assignment Header
% \setlength{\headheight}{33.24544pt}
% \pagestyle{fancy}
% \fancyhead[L]{\bf\large Course \\ Assignment k}
% \fancyhead[R]{\bf\large Keven Qiu}
\parindent=0pt
%\setcounter{section}{0} %
%\begin{flushright}
%Keven Qiu
%\end{flushright}
%
%\author{Keven Qiu}
%\title{\vspace{-15mm}\bf Title\vspace{-5mm}}
%\date{}
%\maketitle
%
%\thispagestyle{empty}
%
%\tableofcontents
%
%\newpage

\section{}

\end{document}

%%%%%%% Insert table %%%%%%%
% \begin{table}[ht]
% \begin{center}
% \begin{tabular}{|c|c|}
%     \hline
%     \hline
% \end{tabular}
% \end{center}
% \end{table}
%%%%%%%%%%%%%%%%%%%%%%%%%%%

%%%%%%% Insert figure %%%%%%%
% \begin{figure}[ht]
%     \centering
%     \includegraphics[width=\textwidth]{} % scale=0.5
% \end{figure}
%%%%%%%%%%%%%%%%%%%%%%%%%%%%%

% \newtcolorbox{tbfbox}[2][]{title=#2,colback=black!5!white,colbacktitle=black!30!white,coltitle=black,arc=1mm,boxrule=1pt,fonttitle=\bfseries}
% \newtcolorbox{df}[2][]{title=Definition: #2,colback=blue!15!white,colbacktitle=blue!30!white,coltitle=black,arc=1mm,boxrule=1pt,fonttitle=\bfseries}
% \newtcolorbox{pp}[2][]{title=Proposition #2,colback=green!15!white,colbacktitle=green!30!white,coltitle=black,arc=1mm,boxrule=1pt,fonttitle=\bfseries}
% \newtcolorbox{thm}[2][]{title= Theorem #2,colback=red!15!white,colbacktitle=red!40!white,coltitle=black,arc=1mm,boxrule=1pt,fonttitle=\bfseries}
% \newtcolorbox{lma}[2][]{title=Lemma #2,colback=red!15!white,colbacktitle=red!30!white,coltitle=black,arc=1mm,boxrule=1pt,fonttitle=\bfseries}
% \newtcolorbox{cor}[2][]{title=Corollary #2,colback=purple!20!white,colbacktitle=purple!40!white,coltitle=black,arc=1mm,boxrule=1pt,fonttitle=\bfseries}
